\documentclass[a4paper, 11pt]{article}

\usepackage{kotex} % Comment this out if you are not using Hangul
\usepackage{fullpage}
\usepackage{hyperref}
\usepackage{amsthm}
\usepackage[numbers,sort&compress]{natbib}

\theoremstyle{definition}
\newtheorem{exercise}{Exercise}

\begin{document}
%%% Header starts
\noindent{\large\textbf{IS-521 Activity Proposal}\hfill
                \textbf{이병학}} \\
         {\phantom{} \hfill \textbf{lbh0307}} \\
         {\phantom{} \hfill Due Date: April 15, 2017} \\
%%% Header ends

\section{Activity Overview}

Secret sharing은 여러 암호화 프로토콜에서 응용되는 것으로, 한 비밀을 여러 인증된
사람에게 나눠주고 그중 일정 이상의 사람이 모였을 때에만 원본 비밀을 알 수 있도록
해주는 알고리즘을 말한다. 이 과제는 학생들이 간단한 형태의 Secret sharing
algorithm을 구현하고 그것의 취약성을 확인한 뒤 발전된 형태의 알고리즘을 구현해보도록
하는 것을 목표로 한다.

\section{Exercises}

\begin{exercise}

  우선, \textbf{Shamir's secret sharing scheme}\cite{SecShareShamir}을 이용해서
  secret을 share하고 recover 하는 프로그램을 제작한다.

\end{exercise}

\begin{exercise}

  위에서 제작한 프로그램을 이용해 Secret sharing을 한 뒤, 이것의 취약성을 이용해
  공격을 해본다. 취약성은 \cite{SecShareShamir}에서 확인할 수 있다.

\end{exercise}

\begin{exercise}

  위의 공격에 안전하도록 더 발전된 Secret sharing\cite{BetterSecShare}을 구현한다.

\end{exercise}

\section{Expected Solutions}

평가는 다음과 같이 이뤄진다.
\begin{enumerate}
  \item 주어진 알고리즘을 잘 구성했다.
  \item Secret을 잘 복원한다.
  \item 충분히 큰 Domain 에 대해서도 잘 작동한다.
\end{enumerate}

\bibliography{references}
\bibliographystyle{plainnat}

\end{document}
